\documentclass[12pt, letterpaper]{article}
\usepackage[a4paper, total={6in, 8in}]{geometry}
\usepackage[utf8]{inputenc}
\usepackage{multirow}
\usepackage[table]{xcolor}

\title{Compte Rendu LO21 \#1}
\author{Philippe Lefebvre, Clea Bordeau, Thomas Habert, Eugène Valty}
\date{Novembre 2021}

\begin{document}

\maketitle

\renewcommand{\contentsname}{Sommaire}
\tableofcontents

\section{Introduction}
Ce premier compte rendu contient un début d'architecture pour le projet, décrivant les différentes classes et leurs interactions.\\
Comme demandé, une liste des différentes tâches ainsi que leur importance est aussi fournie.
\section{Architecture}
\subsection{Introduction}
Vous trouverez ci dessous un \textbf{début} d'architecture; plusieurs notions jugées très importantes pour le sujet n'ont pour l'instant qu'été seulement effleuré en cours \textit{(Hérédité, classe virtuelle, classe générique,...)}. L'architecture est donc voué à être modifiée dans le temps, mais à été réfléchie de façon à ce que chaque modification n'impacte qu'elle même en utilisant le principe d'\textbf{encapsulation}.
\subsection{UML}
\textit{\#insérer uml\#}
\section{Tâches}
\subsection{Introduction}
La liste des tâches inclue essentiellement la construction de l'architecture de base. Il est évident que d'autres tâches \textit{(sur des fonctions spécifiques, des implémentations précises, ...)} viendront s'ajouter au fur et à mesure. On remarquera aussi que les tâches concernant la vue ne sont pas inclues; elles viendront s'ajouter lorsque le modèle sera fonctionnel.
\subsection{Liste des tâches}
\begin{center}
\begin{tabular}{ |c||c||c|p{5cm}| }
\hline
\multicolumn{4}{|c|}{Tâches} \\
\hline
Nom de la tâche & Numéro & Priorité & Prérequis\\
\hline
\hline
Ecrire la classe Jeton & 1 & \cellcolor[HTML]{ee7d7d} Indispensable & Aucun\\
\hline
Ecrire la classe Cartes (générique) & 2 & \cellcolor[HTML]{ee7d7d} Indispensable & Aucun\\
\hline
Ecrire la classe Noble & 3 & \cellcolor[HTML]{eeda7d} Important & classe Cartes\\
\hline
Ecrire la classe Ressources & 4 & \cellcolor[HTML]{eeda7d} Important & classe Cartes\\
\hline
Ecrire la classe Pioche (générique) & 5 & \cellcolor[HTML]{ee7d7d} & Aucun\\
\hline
Ecrire la classe Pioche Nobles & 6 & \cellcolor[HTML]{eeda7d} Important & classe Pioche, classe Noble\\
\hline
Ecrire la classe Pioche Ressources & 7 & \cellcolor[HTML]{eeda7d} Important & classe Pioche, classe Ressources\\
\hline
Ecrire la classe Banque & 8 & \cellcolor[HTML]{ee7d7d} Indispensable & classe Jeton\\
\hline
Ecrire la classe Paquet &  9 & \cellcolor[HTML]{ee7d7d} Indispensable & classe Carte, classe Pioche\\
\hline
Ecrire la classe Plateau & 10 & \cellcolor[HTML]{ee7d7d} Indispensable & classe Carte, classe Pioche, classe Banque\\
\hline
Ecrire la classe Joueur (générique) & 11 & \cellcolor[HTML]{ee7d7d} Indispensable & classe Carte, classe Jeton\\
\hline
Ecrire la classe Joueur Humain & 12 & \cellcolor[HTML]{eeda7d} Important & classe Joueur\\
\hline
Ecrire la classe Joueur IA & 13 & \cellcolor[HTML]{a5ee7d} Utile & classe Joueur\\
\hline
Ecrire la classe Jeu & 14 & \cellcolor[HTML]{ee7d7d} Indispensable & toutes les classes\\
\hline
\end{tabular}
\end{center}
\subsection{Informations sur les tâches}
\begin{tabular}{ |c||c||c|c|c| }
\hline
\multicolumn{5}{|c|}{Tâches} \\
\hline
Nom de la tâche & Numéro & Assigné à & Durée estimée & Avancement\\
\hline
\hline
Ecrire la classe Jeton & 1 & X & X h & 0\%\\
\hline
Ecrire la classe Cartes (générique) & 2 & X & X h & 0\%\\
\hline
Ecrire la classe Noble & 3 & X & X h & 0\%\\
\hline
Ecrire la classe Ressources & 4 & X & X h & 0\%\\
\hline
Ecrire la classe Pioche (générique) & 5 & X & X h & 0\%\\
\hline
Ecrire la classe Pioche Nobles & 6 & X & X h & 0\%\\
\hline
Ecrire la classe Pioche Ressources & 7 & X & X h & 0\%\\
\hline
Ecrire la classe Banque & 8 & X & X h & 0\%\\
\hline
Ecrire la classe Paquet & 9 & X & X h & 0\%\\
\hline
Ecrire la classe Plateau & 10 & X & X h & 0\%\\
\hline
Ecrire la classe Joueur (générique) & 11 & X & X h & 0\%\\
\hline
Ecrire la classe Joueur Humain & 12 & X & X h & 0\%\\
Ecrire la classe Joueur IA & 13 & X & X h & 0\%\\
\hline
Ecrire la classe Jeu & 14 & X & X h & 0\%\\
\hline
\end{tabular}
\end{document}